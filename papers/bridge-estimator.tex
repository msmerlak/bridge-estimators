% ****** Start of file apssamp.tex ******
%
%   This file is part of the APS files in the REVTeX 4.2 distribution.
%   Version 4.2a of REVTeX, December 2014
%
%   Copyright (c) 2014 The American Physical Society.
%
%   See the REVTeX 4 README file for restrictions and more information.
%
% TeX'ing this file requires that you have AMS-LaTeX 2.0 installed
% as well as the rest of the prerequisites for REVTeX 4.2
%
% See the REVTeX 4 README file
% It also requires running BibTeX. The commands are as follows:
%
%  1)  latex apssamp.tex
%  2)  bibtex apssamp
%  3)  latex apssamp.tex
%  4)  latex apssamp.tex
%
\documentclass[%
 reprint,
%superscriptaddress,
groupedaddress,
%unsortedaddress,
%runinaddress,
%frontmatterverbose, 
%preprint,
%preprintnumbers,
%nofootinbib,
%nobibnotes,
%bibnotes,
 amsmath,amssymb,
 aps,
%pra,
%prb,
%rmp,
%prstab,
%prstper,
%floatfix,
]{revtex4-2}

\usepackage{graphicx}% Include figure files
\usepackage{dcolumn}% Align table columns on decimal point
\usepackage{bm}% bold math
%\usepackage{hyperref}% add hypertext capabilities
%\usepackage[mathlines]{lineno}% Enable numbering of text and display math
%\linenumbers\relax % Commence numbering lines

%\usepackage[showframe,%Uncomment any one of the following lines to test 
%%scale=0.7, marginratio={1:1, 2:3}, ignoreall,% default settings
%%text={7in,10in},centering,
%%margin=1.5in,
%%total={6.5in,8.75in}, top=1.2in, left=0.9in, includefoot,
%%height=10in,a5paper,hmargin={3cm,0.8in},
%]{geometry}

\begin{document}


\title{Bridge estimators for birth-death processes}

\author{Aleksander Klimek}
\author{Swadha Pandey}
\author{Matteo Smerlak}
\email{smerlak@mis.mpg.de}
\affiliation{Max Planck Institute for Mathematics in the Sciences, 04103 Leipzig, Germany}


\date{\today}% It is always \today, today,
             %  but any date may be explicitly specified

\begin{abstract}
Whether a birth-death process is likely to grow exponentially over time can be assessed by estimating its basic reproduction number $R_0$. Here we point that out that standard estimators are vulnerable to a selection bias, and propose a technique based on stochastic bridges to correct for this bias. 
\end{abstract}

%\keywords{Suggested keywords}%Use showkeys class option if keyword
                              %display desired
\maketitle

%\tableofcontents

\section{Introduction}

Estimating the probability that a population is likely to grow exponentially over time is a fundamental problem in epidemiology, oncology and other disciplines. That probability is tied to the so-called basic reproduction number $R_0$ of the process, defined as the average number of `offspring' (new infections, new cancer cells) generated by an individual during its lifetime: if $R_0 \leq 1$, the process is sure to eventually go extinct, but if $R_0 > 1$, the probability of explosion is finite. To estimate the reproduction number from empirical data, it is customary to model the growth of the population by a certain birth-death process and use statistical inference to compute a confidence or credible interval for $R_0$.  

This approach is vulnerable to an obvious selection bias: in many applications, population sizes are only reported when they get worryingly large. This means that, whether the generative process is sub- or super-critical, the sample paths on which $R_0$ is estimated usually have an upward trend. In this note we point out that standard estimators become biased under such implicit conditioning, and propose a remedy based on the concept of stochastic bridges. 

A bridge is a process that is conditioned to start and end at prescribed values over a given time interval. Bridges have been studied extensively in the context of discretely observed diffusion models, as a way to simulate missing data. Here we propose to use bridges with a different purpose. Consider a sample path $X = (X_t)$ with length $T$ generated by a birth-death process $\mathcal{X}$ with reproduction number $R_0$. Based on this data, we can propose two different models for $X$. One is the birth-death process itself, obviously. But another possible model is the birth-death bridge $\mathcal{X}^\star$ pinned to the boundary values $X_0$ and $X_T$. Under $\mathcal{X}$, the larger the end value $X_T$, the more likely $R_0$ is to be large. That is not true under $\mathcal{X}^\star$: by construction, the likelihood of $R_0$ (or any other parameter of the model) is independent of $X_T$---it only depends on the shape of the path joining the boundary values $X_0$ and $X_T$. Using the bridge $\mathcal{X}^\star$ instead of the original process $\mathcal{X}$ can therefore protect the estimation of growth parameters from the aforementioned selection bias. 


\section{Bridge estimators}

Consider a continuous-time, homogenenous Markov process $\mathcal{X}$ with transition kernel $p_\theta(x, t; y, s)$, where $\theta$ is a vector of parameters. We assume that $\mathcal{X}$ is observed at discrete times, $X = (X_i, t_i)$. The likelihood function given this data is
\begin{equation}
    \mathcal{L}(\theta\vert X) \propto \prod_{i}p_{\theta}(X_i, t_i; X_{i+1}, t_{i+1}),
\end{equation}
from which one computes the maximum likelihood estimator as $\widehat{\theta} = \textrm{argmax}_\theta\, \mathcal{L}(\theta\vert X)$ (or, alternatively, the Bayesian posterior as $P(\theta\vert X)\propto \mathcal{L}(\theta\vert X) \pi(\theta)$, with $\pi$ a prior). 

A \emph{Markov bridge} is a Markov process conditioned on starting at a value $X_0$ and ending at another value $X_T$ over the duration $T$. Given a sample path $X = (X_t)$ of the process $\mathcal{X}$, we can define an associated bridge $\mathcal{X}^\star$ by considering all paths of $\mathcal{X}$ with identical starting point, ending point and duration as $X$. The transition kernel of this derived process is given by 
\begin{equation}
    p^\star_\theta(x, t; y, s)= \frac{ p_\theta(x, t; y, s)p_\theta(y, s; X_T, T) }{p_\theta(x, t; X_T, T)}.
\end{equation}
Not that, unlike $\mathcal{X}$, the bridge $\mathcal{X}^\star$ is not a homogenous process.\footnote{If $\mathcal{X}$ is a diffusion process with drift $b(x)$ and diffusion coefficient $\sigma(x)$, then $\mathcal{X}^\star$ has drift $b^\star(x) = b(x) + \nabla_x \ln p(X_T, x; T - t)$.} 

Our proposal is to estimate $\theta$ given a sample path $X$ using the likelihood function associated to $\mathcal{X}^\star$ instead of $\mathcal{X}$,  
\begin{equation}
    \mathcal{L}^\star(\theta\vert X) = \frac{\mathcal{L}(\theta\vert X) }{p_\theta(X_0, 0; X_T, T)}.
\end{equation}
In particular, the bridge maximum likelihood estimator is $\widehat{\theta}^\star = \textrm{argmax}_\theta\, \mathcal{L}^\star(\theta\vert X)$, and the bridge posterior is $P^\star(\theta\vert X) \propto \mathcal{L}^\star(\theta\vert X)\pi(\theta)$. By construction, bridge estimators remains unbiased even if $X$ is sampled with a bias towards large ending value $X_T$. 



\end{document}
%


